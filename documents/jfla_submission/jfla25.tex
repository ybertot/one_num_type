\documentclass[draft]{jflart}
\usepackage[utf8]{inputenc}
\usepackage[T1]{fontenc}

% Numéro et année des JFLAs visées par l'article, obligatoire.
\jfla{36}{2025}

\title{Chassez le naturel dans la formalisation des mathématiques}
% Un titre plus court, optionnel.
\titlerunning{Chassez le naturel}

% Auteurs, liste non abrégée.
\author[1]{Yves Bertot}
\author[1]{Thomas Portet}
% Une liste d'auteurs abrégée à utiliser à l'intérieur de l'article.
\authorrunning{Bertot et Portet}

% Affiliations des auteurs
\affil[1]{Centre Inria de l'Université Côte d'Azur, France}

% Une commande définie par l'utilisateur
\newcommand{\cmd}[1]{\texttt{\textbackslash {#1}}}

\begin{document}

\maketitle

\begin{abstract}
  Nous nous intéressons à l'utilisation des systèmes de preuves basés sur la
  théorie des types pour l'enseignement des mathématiques.  Nous voulons
  remettre en question l'approche répandue qui consiste à introduire
  d'abord les nombres naturels, puid d'autres types de nombres.

  Nous explorons une approche ou le type des nombres réels est le seul type
  utilisé pour toutes les définitions concernant des nombres.  Ceci nous oblige
  à re-considérer les outils fournis pour définir des fonctions (en particulier
  la récursion), et pour calculer avec ces fonctions.
  L'une de ces caractéristiques de ce travail est de re-donner de la place à
  la notion d'ensemble dans l'univers de travail.

  Nous illustrons cette approche avec quelques exercices mêlant suites
  récurrentes et nombres réels, où la facilité à manipuler ensemble des nombres
  habituellement séparés dans des types distincts permet des expérimentations
  enrichissantes pour les étudiants.

  L'approche a également des limites que nous tentons de décrire.
\end{abstract}

\section{Introduction}

Il y a maintenant de nombreux systèmes de preuves interactifs
utilisables pour faire des preuves mathématiques.  Même si ces
systèmes de preuves ont été initialement conçus pour vérifier la
correction d'algorithmes et de logiciel, il est souvent pratique
d'inclure des capacités de raisonnement sur des objets mathématiques
pour justifier la correction de logiciels.  Par exemple, les
primitives cryptographiques peuvent reposer sur des propriétés
mathématiques d'objets remarquables comme des courbes elliptiques et
la spécification même de la correction de tels algorithmes peut
reposer sur le concept de probabilité pour un attaquant de contourner
la protection fournie par l'algorithme.

Historiquement, pratiquement tous les systèmes de preuve ont commencé
par décrire les entiers naturels, en spécifiant un type pour cette
catégorie de nombres, puis d'autrse types de nombres sont ajoutés au
fur-et-à-mesure que des concepts de plus en plus avancés sont fournis.
Dans la présentation des données fournies aux utilisateurs finaux, ces
nombres naturels apparaissent donc comme la première donnée
disponible.

Dans les systèmes basés sur la théorie des types avec des types
inductives, comme Agda, Rocq, ou Lean, la situation est renforcée par
le fait que les entiers naturels se décrivent très bien comme un type
inductif.  Cette approche permet de mettre en œuvre des techniques
générales pour disposer de capacités de calcul relativement efficaces.
Ainsi, le calcul d'une expression numérique est fourni directement par
un mécanisme interne appelé conversion, qui permet de remplacer par
une étape de raisonnement des opérations qui seraient représentées par
un grand nombre de pas de réécriture dans d'autres systèmes de preuve.

Lorsque l'on fait des mathématiques plus avancées, on est amené à
utiliser d'autres types de nombres, en particulier une grande partie
des mathématiques étudiées à l'école et dans les premières années
universitaires repose sur les nombres réels.  Si l'on veut couvrir
dans la bibliothèque d'un système de preuve des connaissances
correspondant à ce programme, on est naturellement amené à écrire des
formules où nombres naturels et nombre réels se cotoient.

L'essor des systèmes de preuve interactifs est tel que l'on peut
maintenant s'interroger sur leur apport dans l'enseignement des
mathématiques.  L'apprentissage des mathématiques contient plusieurs
aspects.  Dans les premières étapes, l'étudiant doit apprendre à
calculer.  Pendant des générations, les élèves ont dû apprendre à
maitriser les algorithmes d'addition, de multiplication, et de
division pour les nombres décimaux avec partie fractionnaire.  Cet
apprentissage a été remis en question dans les dernières décennies à
cause de l'apparition des machines à calculer.  L'un des effets de
bord de cette apparition est que les moins courageux des élèves sont
enclins à refuser de s'investir dans cet apprentissage, parce que
l'existence des machines calculés rend se savoir-faire inutile.

Dans une deuxième étape, l'étudiant en mathématique doit apprendre à
raisonner.  Pour raisonner, il faut savoir appliquer des
syllogismes, faire la différence entre les hypothèses et la conclusion
d'une phrase, comprendre des phrases énonçant l'existence d'un objet
ou des phrases énonçant qu'une propriété est universellement
satisfaite, et faire la difference entre ces deux types de phrases.

Pour de nombreux étudiants en mathématiques, le langage à utiliser est
une langue étrangère.  Il faut pratiquer cette langue étrangère
fréquemment pour progresser.  En particulier, il faut apprendre à décrire
des raisonnements en suivant correctement les implications fournies par
les théorèmes fournis, et en faisant correctement les transformations
de formules permises par les égalités connues.  En fin de compte,
l'apprentissage est réussi si l'étudiant est capable de faire un
raisonnement qui sera considéré comme correct par une lectrice humaine.
En d'autres termes, il s'agit d'apprendre le langage qui permet de
s'inclure dans la communauté humaine des mathématiciens.

Un système de preuve interactif est justement conçu pour vérifier que toutes
les étapes de raisonnement sont effectuées correctement.  Il est donc
naturel de vouloir explorer l'utilisation d'un tel système de preuve
pour l'enseignement des mathématiques.  En revanche, il faut éviter
que ce nouvel outil impose une charge d'apprentissage supplémentaire,
et finalement contreproducive si l'étudiant passe trop de temps à
s'adapter au système de preuve, comme une informaticienne apprend un
langage de programmation, mais n'a plus le temps d'apprendre les
concepts mathématiques de son cursus.  Il faut éviter que le langage
mathématique imposé par le système de preuve soit trop éloigné du
langage usuel pour le niveau d'apprentissage de l'étudiant.

Les nombres naturels, tels que manipulés dans les systèmes de preuve,
sont fournis comme un type inductif.  La théorie des types inductifs
fournit la possibilité de définir des fonctions, de telle que sorte
que les opérations de base peuvent être définies, comme l'addition et
la multiplication.  On peut également envisager la soustraction et la
division, mais c'est un point où apparait une dissonance entre les
mathématiques telle qu'elles apparaissent dans le système de preuve et
dans la tradition humaine.  Pour pouvoir composer les opérations à
l'envi, il est préférable que le résultat d'une soustraction de deux
nombres naturels soit un nombre naturel, le système de preuve impose
que la fonction soit totale, donc le concepteur de la bibliothèque est
obligé de donner une valeur à la formule \(3 - 5\).  Le choix fait
dans tous les systèmes de preuve est que le résultat est le nombre
\(0\), parce que l'autre valeur envisageable \(-2\) n'est pas un
entier naturel.

La tradition des mathématiques et d'associer à chaque fonction son
domaine de définition.  Lorsque l'on écrit une formule composée, on
doit expliciter le fait que toutes les fonctions sont bien utilisées
dans leur domaine de définition.  La tradition des mathématiques
formalisées en théorie des types est que toute fonction doit être
totale.  Les fonctions donc généralement prolongées avec une valeur
par défaut aux endroits ou leur correspondantes traditionnelles sont
non définies.  La modélisation reste fidèle dans la mesure où les
théorèmes qui permettent de raisonner sur la fonction utilisent
généralement l'appartenance des arguments au domaine de définition
comme hypothèse.  Ainsi, une mathématicienne utilisera naturellement
le raisonnement:
\[(m - n) + n = m\]
en reposant sur le fait que la formule de droite a été validée (\(n
\leq m\)) plus tôt dans le discours.  En revanche, l'outil de preuve
fournit un théorème avec l'énoncé suivant:
\begin{verbatim}
forall m n, n <= m -> (m - n) + n = m
\end{verbatim}
Ainsi, la condition de bonne formation de la formule n'est pas
vérifiée a priori, mais uniquement au moment où l'on veut exploiter
les propriétés de cette fonction de soustraction.  Les bonnes
vérifications sont faites, mais au bon moment.

Il reste que l'utilisatrice finale doit mémoriser le fait que
lorsqu'elle voit l'opération \(m - n\) dans une formule mathématique,
il ne s'agit pas de la soustraction usuelle, mais de cette
soustraction si \(n\) est plus petit que \(m\) et 0 sinon.

Le coût pour l'utilisatrice finale n'est pas seulement un coût de
mémorisation.  Les outils de démonstration (appelés des tactiques)
fournis dans le système de preuve sont également impactés par la
présence de cette représentation irrégulière de la soustraction.
Dans le système Coq, il existe une tactique appelée \texttt{ring}, qui
est capable de prouver des égalités entre formules polynomiales, en
exploitant les propriétés des différents opérateurs.  Cette tactique
est très pratique et son comportement très facile à comprendre, sauf
qu'elle est ``bloquée'' en présence de soustractions sur les entiers
naturels.  Par exemple, la formule suivante est une formule évidente à
l'œil nu que cette tactique ne sait pas résoudre.
\[(n + m - (m + n)) * n = 0\]

Quand on avance dans les mathématiques, on est de toutes façon amené à
étudier des ensembles de nombres plus riches que l'ensemble des
entiers naturels.  Par exemple, si l'on considère la suite de
Fibonacci \({\mathcal F}\), définie par les équations suivantes:
\[{\mathcal F}(0) = 0 \qquad {\mathcal F}(1)= 1\qquad {\mathcal F}(n +
2) = {\mathcal F}(n) + {\mathcal F}(n + 1),\]
un résultat mathématique connu est donné par la formule suivante:
\[{\mathcal F} (n) = \frac{\phi ^n - ({\frac{-1}{\phi}}) ^
  n}{\sqrt{5}},\]
où \(\phi\) est le nombre d'or.
\[\phi = \frac{\sqrt{5} + 1}{2}\]
Alors que le membre gauche de l'égalité fait intervenir une fonction défínie
seulement par récurrence sur les entiers naturels et en utilisant des
additions de nombres naturels, la formule de gauche fait intervenir des
opérations qui ne sont pas stables pour les nombres naturels, les nombres
entiers relatifs, ou même les nombres rationnels.

Nous proposons de développer une libraire d'initiation à la preuve
mathématique dans laquelle on n'utilise qu'un seul type de nombres,
pour que les étudiants puissent travailler confortablement avec les
concepts mathématiques qu'ils doivent apprendre à maitriser.  Ce
travail est complémentaire des efforts proposés par d'autres
chercheurs pour fournir un environnement où les étudiants peuvent
bénéficier d'une approche leur permettant de rédiger leurs preuves
dans un style de ``langue naturelle controlée'' parce que ces efforts
visent à améliorer le langage utilisés pour décrire les raisonnements
logiques, sans modifier le cadre de travail; les faits utilisés
restent proches de la théorie des types traditionnelle et reposent sur
plusieurs types de nombres.

Au fur et à mesure que les étudiants acquièrent la maitrise des
mathématiques et potentiellement des systèmes de preuve interactifs,
il est probablement justifié de revenir à la présentation usuelle des
nombres, reposant sur plusieurs types.  En particulier, nous verrons
que les types inductifs de nombre jouent un rôle crucial dans
l'implémentation des différents fonctionnalités que nous proposons.

Dans notre bibliothèque, les nombres naturels sont présents, mais
présentés comme un sous-ensemble des nombres réels.  De même nous
fournissons un ensemble des nombres entiers (et dans une version plus
complète de la librarie, nous proposerons certainement aussi un
ensemble des nombres rationnels).  L'ensemble des nombres naturels est
décrit par un prédicat inductif de type \({\mathbb R} \rightarrow
{\mathbb Prop}\).  Ceci nous permet de bénéficier directement des
fonctionnalités existantes pour les preuves par récurrence.  En
revanche, cette approche ne nous permet pas de définir des fonctions
récursives sur les entiers naturels.  Nous fournissons une commande
pour remplir cette lacune, et nous montrons sur quelques exemples que
cette commande de retrouver la capacité de raisonner sur des suites
récurrentes, un sujet qui est souvent abordé dans les premiers de
cours de mathématiques.

Dans un premier temps, nous décrirons les grandes lignes d'implémentation des
différentes commandes que nous fournissons pour donner au type des
nombres réels plusieurs des fonctionalités qui étaient normalement
réservées aux types inductifs, nous donnons quelques exemples qui
illustrent les gains que permettent cette approche, en terme de
simplicité d'utilisation pour un étudiant découvrant à la fois les
mathématiques et les systèmes de preuve interactifs.

Ensuite nous montrerons comment quelques exercices de mathématiques
peuvent être réalisés dans ce cadre, en soulignant comment cette
approche permet aux étudiants de rester proches de l'univers
mathématiques qui devrait leur être familier.
\section{Décrire des sous-ensembles des nombres réels}
Dans le système de preuve, l'ensemble des nombres réels \(\mathcal R\)
est représenté par un type dénoté \texttt{R}.  Les opérations usuelles
sur les nombres sont représentées par des fonctions \texttt{+}, \texttt{-},
\texttt{*}, \texttt{/}.  Les constantes entières de type \texttt{R} s'écrivent
directement dans leur représentation décimale.  De cette manière, une
utilisatrice finale peut se contenter d'écrire le texte suivant:
\begin{verbatim}
3 - 5 * 2
\end{verbatim}
Nous reviendrons plus tard sur le dispositif qui permet cette
utilisation de notations décimales, car il repose sur des nombres
entiers dans un type distinct, d'une façon dont l'utilisatrice ne
devrait pas avoir conscience.

L'approche usuelle pour définir un ensemble dans le système preuve
Rocq est d'identifier l'ensemble avec la propriété caractéristique de
cet ensemble.  Cette propriété caractéristique est une fonction, de
telle sorte que l'on écrit \texttt{Rnat n} pour exprimer que \texttt{n} est
dans l'ensemble \texttt{Rnat}.  Le système de preuve est basé sur la
théorie des types, donc cette fonction doit avoir un type de départ
(où elle prend ses arguments).  Ici, nous considérons des
sous-ensembles de \(\mathcal R\) donc la fonction a le type \texttt{R ->
  Prop}.

Les nombres naturels peuvent être décrits comme le sous-ensemble
minimal de \(\mathcal R\) qui contient \(0\) et qui est stable par
l'operation d'ajouter \(1\).  Le fait que l'ensemble considéré soit
minimal donne naturellement l'existence d'un principe de récurrence
qui s'énonce de la façon suivante: {\em toute propriété qui est vraie
  pour \(0\) et qui est vraie pour tout nombre de la forme \(n + 1\)
  si elle est déjà vraie pour \(n\) est vraie pour tous les entiers
  naturels}.

Dans les systèmes de preuve, une telle propriété défine par minimalité
peut généralement être construire à l'aide d'une définition inductive.
Pour les nombres naturels, la voici:
\begin{verbatim}
Inductive Rnat : R -> Prop :=
| Rnat0 : Rnat 0
| Rnat_suc : forall n, Rnat n -> Rnat (n + 1).
\end{verbatim}

Grâce à cette définition, on prouve aisément que la somme de deux
nombres naturels est un nombre naturel (par récurrence sur l'un des
deux arguments), de même pour le produit.  En revanche la soustraction
de deux nombres naturels ne satisfait pas nécessairement cette
propriété.  La différence de deux nombres est un entier relatif, et
cet entier relatif est un nombre naturel sous une condition d'ordre.

Ceci nous amène naturellement à considérer le sous-ensemble de \texttt{R}
qui contient les entiers relatifs.  Plusieurs façons de définir les
entiers positifs sont disponibles, nous avons choisi la suivante:
\begin{verbatim}
Inductive Rint : R -> Prop :=
| Rint1 : Rint 1
| Rint_sub : forall x y, Rint x -> Rint y -> Rint (x - y).
\end{verbatim}
Cette définition inductive permet de montrer que la somme de deux
nombres entiers est un nombre entiers, de même pour le produit et
la soustraction.

L'utilisatrice finale ne sera probablement pas amenée à utiliser le
principe de récurrence associé à la définition inductive \texttt{Rint}.
En revanche les preuves par récurrences usuelles d'un cours de
mathématiques seront habituellement faisable directement avec le
principe de récurrence associé avec \texttt{Rnat}.

Enfin, l'utilisatrice finale ne devra pas le voir, mais en revanche
l'éducatrice ou la personne en charge de développer la bibliothèque
fournie à l'utilisatrice finale pourra avoir besoin d'établir une
correspondance entre l'ensemble \texttt{Rnat} et le type inductif
\texttt{nat}.  Nous aurons besoin de cette corespondance pour définir des
suites récurrentes de nombres réels.

La bibliothèque standard des nombres réels de Coq fournit une fonction
récursive \texttt{INR} de type \texttt{nat -> R}.  Cette fonction est
définie récursivement, de telle sorte que l'on peut montrer facilement
que l'image de \texttt{INR} coincide avec l'ensemble \texttt{Rnat}.  Les
deux énoncés sont les suivants:
\begin{verbatim}
Rnat_INR n : Rnat (INR n).

Rnat_exists_nat x {xnat : Rnat x} :
  exists n, x = IZR (Z.of_nat n).
\end{verbatim}

Les différents théorèmes de stabilité de l'ensemble des nombres naturels
sont déclarés comme des instances de classes de types, de telle sorte que
les preuves qu'une formule mathématique est bien un nombre naturel peuvent
souvent être effectuées automatiquement.

Un des éléments remarquables de ces théorèmes de stabilité est le théorème
qui exprime que toute constante numériques comme \(0\), \(3\), \(42\) est
automatiquement prouvée comme appartenant à l'ensemble \texttt{Rnat}.

En revanche, le théorème qui permet de prouver qu'une soustraction retourne un
nombre naturel a la forme suivante:
\begin{verbatim}
Rnat_sub : forall n m, Rnat n -> Rnat m -> m <= n -> Rnat(n - m)
\end{verbatim}
Ce théorème doit souvent être invoqué interactivement.
\section{Définitions Récursives de Fonctions}
L'une des caractéristques essentielles des types inductifs dans les
systèmes de preuve est qu'ils permettent de définir des fonctions
récursives.  En décidant de représenter tous les nombres utilisés dans
le discours mathématique par des nombres réels, nous perdons cette
caractéristiques, parce que les nombres réels ne sont pas définis par
un type inductif (et ils ne peuvent pas l'être).

Pour remédier à cela, nous fournissons une commande qui permet de
définir des fonctions de \texttt{R} dans \texttt{R} de telle sorte que leur
valeur est spécifiée précisément pour tous les nombres naturels par un
procédé récursif et {\em indéfinie} ailleurs.  Cet usage de valeurs
indéfinie est très similaire à ce qui se produit pour l'inverse, dont
la valeur n'est pas définie en 0.  Pour des commodités d'écriture,
l'inverse de 0 peut être manipulé comme un nombre réel, mais aucune
propriété ne peut obtenue pour ce nombre réel.

Donnons un exemple de définition obtenue par notre commande:
\begin{verbatim}
Recursive (def factorial such that factorial 0 = 1 /\
  forall n, Rnat (n - 1) -> factorial n = n * factorial (n - 1)).
\end{verbatim}
À l'exécution de cette commande, deux nouveaux objets Coq sont
engendrés, le premier est une fonction nommée \texttt{factorial} de
type \texttt{R -> R} et le
deuxième est un théorème qui reprend exactement la conjonction décrite
ci-dessus.

Il est évident que la spécification donnée ci-dessus ne permet pas de
décrire la valeur de \texttt{factorial} pour l'entrée \texttt{0.5}.
En revanche, elle permet bien d'obtenir le résultat que
\texttt{factorial 3} est égal à \texttt{6}, en reposant sur plusieurs
appels au deuxième membre de la conjonction, et à aux preuves que
\texttt{3 - 1}, \texttt{3 - 1 - 1}, \texttt{3 - 1 - 1 - 1} sont des
nombres naturels, et à la vérification que \texttt{3 * (3 - 1) * (3 -
  1 - 1) * (3 - 1 - 1 - 1) = 6}.  Cette égalité est prouvée en une
seule étape par la tactique \texttt{ring}, parce que les nombres
considérés sont des nombres réels et la tactique \texttt{ring} n'a pas de
restriction vis-à-vis de la soustraction pour ce type de nombre.

La commande de définition de suite récurrentes que nous avons conçue permet
également de définir des suites récurrentes d'ordre arbitrairement élevé
(mais fini).  Par exemple,  la suite de Fibnoacci peut se définir de la façon
suivante (nous nommons la fonction \texttt{fib} pour gagner de la place).

\begin{verbatim}
Recursive (def fib such that 
    fib 0 = 0 /\ 
    fib 1 = 1 /\
    forall n : R, Rnat (n - 2) -> 
    fib n = fib (n - 2) + fib (n - 1)).
\end{verbatim}
La personne qui définit une nouvelle fonction récursive peut donc
mettre autant de case de base qu'elle le désire, et une specification
quantifiée universellement qui utilise un nombre de valeurs
précédentes égal au nombre de valeurs de base.

\section{Techniques d'implémentation}
Ce que nous allons décrire dans cette section est la méthode que nous avons
employée pour concevoir notre commande de définition récursive.
L'utilisatrice finale n'a pas besoin de connaitre ces détails: les
seuls éléments mis à sa disposition sont la fonction définie avec son
type, et le théorème qui collecte la spécification de comportement
sous forme d'une conjonction d'égalités, dont une est quantifiée
universellement sur un nombre dont un prédécesseur doit être un nombre
naturel.

Puisque l'utilisatrice finale ne voit pas ce code, nous avons le droit
d'utiliser les nombres naturels, et nous pouvons utiliser un outil de
méta-programmation pour manipuler la spécification fournie par
l'utilisateur et construire une définition répondant à la
sécification.

\section{Conditions générales JFLA}
%
En particulier, l'usage de paquets comme~\texttt{times} ou~\texttt{libertine}
est interdit.
%
Modifier les paramètres typographiques qui régissent l'apparence du code
informatique ou des formules mathématiques pour faire tenir un contenu trop
dense dans une zone trop étroite est également de mauvais aloi.

L'usage des commandes de positionnement et d'espacement explicites,
notamment~\cmd{vspace} et ses cousines, doit être minimisé.
%
La commande~\cmd{sloppy} ne doit être utilisée qu'en ultime recours,
lorsqu'aucune reformulation du texte n'est possible.

L'option~\texttt{draft} doit impérativement être utilisée lors de la production
de la version soumise pour relecture.
%
Elle devra être remplacée par l'option~\texttt{final} pour la version finale.

\section{Options de la classe}

\paragraph{Option~\texttt{english}.}

La classe suppose par défaut un article rédigé en langue française.
%
Cette option est à utiliser si l'article est rédigé en langue anglaise.

\paragraph{Option~\texttt{draft}.}

Cette option rajoute les numéros de lignes dans la marge.

\paragraph{Option~\texttt{final}.}

L'option~\texttt{final} prépare l'article à l'inclusion dans les actes de la
conférence.

\section{Paquets chargés par défaut}

La classe~\texttt{jflart.cls} charge un certain nombre de paquets par défaut.
%
\begin{itemize}
\item
  %
  Le paquet~\texttt{babel} pour la prise en charge de la langue française ou
  anglaise.

\item
  %
  Les paquets~\texttt{color} et~\texttt{graphicx}, pour permettre l'usage de
  couleurs et l'inclusion d'images.

\item
  %
  Le paquet~\texttt{hyperref}, pour l'ajout des hyperliens.
  %
  Ceux-ci, par défaut, ne sont pas mis en surbrillance afin de préserver le gris
  typographique du texte.

\item
  %
  Les paquets de l'\emph{American Mathematical Society}, à
  savoir~\texttt{amsmath}, \texttt{amssymb} et~\texttt{amsthm}.
  %
  Nous recommandons vivement l'usage des environnements proposés par ces paquets
  pour énoncer théorèmes, lemmes et définitions (voir plus bas), ainsi que pour
  aligner d'éventuelles équations et
  formules~(environnements~\texttt{align},~\texttt{aligned},~\texttt{cases},
  etc.).

\item
  %
  Le paquet~\texttt{marthpartir} de D.~Rémy, qui propose un support natif pour
  les mathématiques en mode paragraphe ainsi qu'une commande pour les règles
  d'inférence.

\end{itemize}

\section{Divers}

\subsection{Figures}

\begin{figure}
  \centering
  \includegraphics[width=.8\linewidth]{jfla.jpg}
  \caption{Les JFLA 2002, photographie par Maxence Guesdon}
  \label{fig:bienbelle}
\end{figure}

L'usage de~\texttt{includegraphics} avec une option~\texttt{width} permet une
utilisation facile d'images, comme illustré par la figure~\ref{fig:bienbelle}.

\subsection{Mathématiques}

Les environnements suivants ont été prédéfinis via le paquet~\texttt{amsthm}.

\begin{center}
  \begin{tabular}{lll}
    Environnement & Nom français & Nom anglais
    \\
    \hline
    \texttt{theo} & Théorème & \emph{Theorem}
    \\
    \texttt{prop} & Proposition & \emph{Proposition}
    \\
    \texttt{conj} & Conjecture & \emph{Conjecture}
    \\
    \texttt{coro} & Corollaire & \emph{Corollary}
    \\
    \texttt{lemm} & Lemme & \emph{Lemma}
    \\
    \texttt{defi} & Définition & \emph{Definition}
    \\
    \texttt{rema} & Remarque & \emph{Remark}
    \\
    \texttt{exem} & Exemple & \emph{Example}
  \end{tabular}
\end{center}

\subsection{Code source}

La classe~\texttt{jflart.cls} ne propose pas de paquet pour la coloration
syntaxique par défaut.
%
Les paquets~\texttt{listings} et~\texttt{minted} sont les plus fréquemment
utilisés.

\subsection{Bibliographie}

Vous pouvez au choix utiliser bibtex ou biblatex pour gérer votre bibliographie.
%
Merci d'utiliser le style de citation~\texttt{alpha-fr} avec bibtex, ou son
équivalent biblatex.

\subsection{Remerciements}

La classe ne propose pas d'environnement dédié pour les remerciements et sources de financement éventuelles.
%
Nous vous suggérons d'utiliser la commande~\cmd{paragraph\{Remerciements.\}} en
fin d'article.

\subsection{Autres ressources}

La classe~\texttt{jflart.cls} et sa documentation sont disponibles en
ligne~\cite{JFLART}.

\bibliographystyle{alpha-fr}
\bibliography{jflart}

\end{document}
