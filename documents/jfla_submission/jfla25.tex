\documentclass[draft]{jflart}
\usepackage[utf8]{inputenc}
\usepackage[T1]{fontenc}

% Numéro et année des JFLAs visées par l'article, obligatoire.
\jfla{36}{2025}

\title{Chassez le naturel dans la formalisation des mathématiques}
% Un titre plus court, optionnel.
\titlerunning{Chassez le naturel}

% Auteurs, liste non abrégée.
\author[1]{Yves Bertot}
\author[1]{Thomas Portet}
% Une liste d'auteurs abrégée à utiliser à l'intérieur de l'article.
\authorrunning{Bertot et Portet}

% Affiliations des auteurs
\affil[1]{Centre Inria de l'Université Côte d'Azur, France}

% Une commande définie par l'utilisateur
\newcommand{\cmd}[1]{\texttt{\textbackslash {#1}}}

\begin{document}

\maketitle

\begin{abstract}
  Nous nous intéressons à l'utilisation des systèmes de preuves basés sur la
  théorie des types pour l'enseignement des mathématiques.  Nous voulons
  remettre en question l'approche répandue qui consiste à introduire
  d'abord les nombres naturels, puid d'autres types de nombres.

  Nous explorons une approche ou le type des nombres réels est le seul type
  utilisé pour toutes les définitions concernant des nombres.  Ceci nous oblige
  à re-considérer les outils fournis pour définir des fonctions (en particulier
  la récursion), et pour calculer avec ces fonctions.
  L'une de ces caractéristiques de ce travail est de re-donner de la place à
  la notion d'ensemble dans l'univers de travail.

  Nous illustrons cette approche avec quelques exercices mêlant suites
  récurrentes et nombres réels, où la facilité à manipuler ensemble des nombres
  habituellement séparés dans des types distincts permet des expérimentations
  enrichissantes pour les étudiants.

  L'approche a également des limites que nous tentons de décrire.
\end{abstract}

\section{Introduction}

Il y a maintenant de nombreux systèmes de preuves interactifs
utilisables pour faire des preuves mathématiques.  Même si ces
systèmes de preuves ont été initialement conçus pour vérifier la
correction d'algorithmes et de logiciel, il est souvent pratique
d'inclure des capacités de raisonnement sur des objets mathématiques
pour justifier la correction de logiciels.  Par exemple, les
primitives cryptographiques peuvent reposer sur des propriétés
mathématiques d'objets remarquables comme des courbes elliptiques et
la spécification même de la correction de tels algorithmes peut
reposer sur le concept de probabilité pour un attaquant de contourner
la protection fournie par l'algorithme.

Historiquement, pratiquement tous les systèmes de preuve ont commencé
par décrire les entiers naturels, en spécifiant un type pour cette
catégorie de nombres, puis d'autrse types de nombres sont ajoutés au
fur-et-à-mesure que des concepts de plus en plus avancés sont fournis.
Dans la présentation des données fournies aux utilisateurs finaux, ces
nombres naturels apparaissent donc comme la première donnée
disponible.

Dans les systèmes basés sur la théorie des types avec des types
inductives, comme Agda, Rocq, ou Lean, la situation est renforcée par
le fait que les entiers naturels se décrivent très bien comme un type
inductif.  Cette approche permet de mettre en œuvre des techniques
générales pour disposer de capacités de calcul relativement efficaces.
Ainsi, le calcul d'une expression numérique est fourni directement par
un mécanisme interne appelé conversion, qui permet de remplacer par
une étape de raisonnement des opérations qui seraient représentées par
un grand nombre de pas de réécriture dans d'autres systèmes de preuve.

Lorsque l'on fait des mathématiques plus avancées, on est amené à
utiliser d'autres types de nombres, en particulier une grande partie
des mathématiques étudiées à l'école et dans les premières années
universitaires repose sur les nombres réels.  Si l'on veut couvrir
dans la bibliothèque d'un système de preuve des connaissances
correspondant à ce programme, on est naturellement amené à écrire des
formules où nombres naturels et nombre réels se cotoient.

L'essor des systèmes de preuve interactifs est tel que l'on peut
maintenant s'interroger sur leur apport dans l'enseignement des
mathématiques.  L'apprentissage des mathématiques contient plusieurs
aspects.  Dans les premières étapes, l'étudiant doit apprendre à
calculer.  Pendant des générations, les élèves ont dû apprendre à
maitriser les algorithmes d'addition, de multiplication, et de
division pour les nombres décimaux avec partie fractionnaire.  Cet
apprentissage a été remis en question dans les dernières décennies à
cause de l'apparition des machines à calculer.  L'un des effets de
bord de cette apparition est que les moins courageux des élèves sont
enclins à refuser de s'investir dans cet apprentissage, parce que
l'existence des machines calculés rend se savoir-faire inutile.

Dans une deuxième étape, l'étudiant en mathématique doit apprendre à
raisonner.  Pour raisonner, il faut savoir appliquer des
syllogismes, faire la différence entre les hypothèses et la conclusion
d'une phrase, comprendre des phrases énonçant l'existence d'un objet
ou des phrases énonçant qu'une propriété est universellement
satisfaite, et faire la difference entre ces deux types de phrases.

Pour de nombreux étudiants en mathématiques, le langage à utiliser est
une langue étrangère.  Il faut pratiquer cette langue étrangère
fréquemment pour progresser.



\section{Consignes générales}

Pour maintenir l'uniformité des actes des JFLA, nous vous demandons de ne
changer ni la police par défaut ni sa taille (10 points).
%
En particulier, l'usage de paquets comme~\texttt{times} ou~\texttt{libertine}
est interdit.
%
Modifier les paramètres typographiques qui régissent l'apparence du code
informatique ou des formules mathématiques pour faire tenir un contenu trop
dense dans une zone trop étroite est également de mauvais aloi.

L'usage des commandes de positionnement et d'espacement explicites,
notamment~\cmd{vspace} et ses cousines, doit être minimisé.
%
La commande~\cmd{sloppy} ne doit être utilisée qu'en ultime recours,
lorsqu'aucune reformulation du texte n'est possible.

L'option~\texttt{draft} doit impérativement être utilisée lors de la production
de la version soumise pour relecture.
%
Elle devra être remplacée par l'option~\texttt{final} pour la version finale.

\section{Options de la classe}

\paragraph{Option~\texttt{english}.}

La classe suppose par défaut un article rédigé en langue française.
%
Cette option est à utiliser si l'article est rédigé en langue anglaise.

\paragraph{Option~\texttt{draft}.}

Cette option rajoute les numéros de lignes dans la marge.

\paragraph{Option~\texttt{final}.}

L'option~\texttt{final} prépare l'article à l'inclusion dans les actes de la
conférence.

\section{Paquets chargés par défaut}

La classe~\texttt{jflart.cls} charge un certain nombre de paquets par défaut.
%
\begin{itemize}
\item
  %
  Le paquet~\texttt{babel} pour la prise en charge de la langue française ou
  anglaise.

\item
  %
  Les paquets~\texttt{color} et~\texttt{graphicx}, pour permettre l'usage de
  couleurs et l'inclusion d'images.

\item
  %
  Le paquet~\texttt{hyperref}, pour l'ajout des hyperliens.
  %
  Ceux-ci, par défaut, ne sont pas mis en surbrillance afin de préserver le gris
  typographique du texte.

\item
  %
  Les paquets de l'\emph{American Mathematical Society}, à
  savoir~\texttt{amsmath}, \texttt{amssymb} et~\texttt{amsthm}.
  %
  Nous recommandons vivement l'usage des environnements proposés par ces paquets
  pour énoncer théorèmes, lemmes et définitions (voir plus bas), ainsi que pour
  aligner d'éventuelles équations et
  formules~(environnements~\texttt{align},~\texttt{aligned},~\texttt{cases},
  etc.).

\item
  %
  Le paquet~\texttt{marthpartir} de D.~Rémy, qui propose un support natif pour
  les mathématiques en mode paragraphe ainsi qu'une commande pour les règles
  d'inférence.

\end{itemize}

\section{Divers}

\subsection{Figures}

\begin{figure}
  \centering
  \includegraphics[width=.8\linewidth]{jfla.jpg}
  \caption{Les JFLA 2002, photographie par Maxence Guesdon}
  \label{fig:bienbelle}
\end{figure}

L'usage de~\texttt{includegraphics} avec une option~\texttt{width} permet une
utilisation facile d'images, comme illustré par la figure~\ref{fig:bienbelle}.

\subsection{Mathématiques}

Les environnements suivants ont été prédéfinis via le paquet~\texttt{amsthm}.

\begin{center}
  \begin{tabular}{lll}
    Environnement & Nom français & Nom anglais
    \\
    \hline
    \texttt{theo} & Théorème & \emph{Theorem}
    \\
    \texttt{prop} & Proposition & \emph{Proposition}
    \\
    \texttt{conj} & Conjecture & \emph{Conjecture}
    \\
    \texttt{coro} & Corollaire & \emph{Corollary}
    \\
    \texttt{lemm} & Lemme & \emph{Lemma}
    \\
    \texttt{defi} & Définition & \emph{Definition}
    \\
    \texttt{rema} & Remarque & \emph{Remark}
    \\
    \texttt{exem} & Exemple & \emph{Example}
  \end{tabular}
\end{center}

\subsection{Code source}

La classe~\texttt{jflart.cls} ne propose pas de paquet pour la coloration
syntaxique par défaut.
%
Les paquets~\texttt{listings} et~\texttt{minted} sont les plus fréquemment
utilisés.

\subsection{Bibliographie}

Vous pouvez au choix utiliser bibtex ou biblatex pour gérer votre bibliographie.
%
Merci d'utiliser le style de citation~\texttt{alpha-fr} avec bibtex, ou son
équivalent biblatex.

\subsection{Remerciements}

La classe ne propose pas d'environnement dédié pour les remerciements et sources de financement éventuelles.
%
Nous vous suggérons d'utiliser la commande~\cmd{paragraph\{Remerciements.\}} en
fin d'article.

\subsection{Autres ressources}

La classe~\texttt{jflart.cls} et sa documentation sont disponibles en
ligne~\cite{JFLART}.

\bibliographystyle{alpha-fr}
\bibliography{jflart}

\end{document}
