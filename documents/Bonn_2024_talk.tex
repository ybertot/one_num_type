\documentclass[compress]{beamer}
\usepackage[latin1]{inputenc}
\usepackage{alltt}
\newdimen\topcrop
\topcrop=10cm  %alternatively 8cm if the pdf inclusions are in letter format
\newdimen\topcropBezier
\topcropBezier=19cm %alternatively 16cm if the inclusions are in letter format

\setbeamertemplate{footline}[frame number]
\title{A single number type for Math education in Type Theory}
\author{Yves Bertot}
\date{June 2024}
\mode<presentation>
\begin{document}

\maketitle
\begin{frame}
\frametitle{The context}
\begin{itemize}
\item Type theory base theorem provers are increasingly used for mathematics
\item Mathematical Components, Mathlib, Isabelle's AFP, etc,
are directed at experts
\item Claim: these libraries are not adapted to provide help in teaching
\begin{itemize}
\item Strong inspiration: Waterproof (and similar experiments with controled natural language)
\item The language is not the only problem, the material may also be an issue
\item Strong contention again the natural numbers
\end{itemize}
\end{itemize}
\end{frame}
\begin{frame}
\frametitle{Issues with the natural numbers}
\begin{itemize}
\item Positive sides
\begin{itemize}
\item An inductive type
\item computation by reduction (faster than rewriting)
\item Proof by induction as instance of a general scheme
\item Recusive definitions are quite natural
\end{itemize}
\item Negative sides
\begin{itemize}
\item Subtraction is odd: the value of \(3-5\) is counterintuitive
\item The status of function {\tt S} is difficult to grasp.
\item Too much cognitive load for struggling students
\end{itemize}
\end{itemize}
\end{frame}
\end{document}


%%% Local Variables: 
%%% mode: latex
%%% TeX-master: t
%%% End: 
