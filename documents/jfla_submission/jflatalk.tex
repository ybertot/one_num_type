\documentclass[compress]{beamer}
\usepackage[french]{babel}
\usepackage[latin1]{inputenc}
\usepackage[T1]{fontenc}
\usepackage{cancel}
\usepackage{alltt}
\usepackage{graphicx}
\graphicspath{ {./images/} }

\newdimen\topcrop
\topcrop=10cm  %alternatively 8cm if the pdf inclusions are in letter format
\newdimen\topcropBezier
\topcropBezier=19cm %alternatively 16cm if the inclusions are in letter format

\setbeamertemplate{footline}[frame number]
\title{Chassez le naturel\\ dans la formalisation des math\'ematiques\\
}
\author{Thomas Portet}
\date{JFLA Janvier 2025
\newline
\newline\newline
\newline
\includegraphics[width=2cm]{Inr_logo_rouge.svg}\\[2cm] }
\mode<presentation>
\begin{document}
% Nous affirmons que les tactiques des outils de preuves suivent une certaine logique et nous voulons que cette logique puisse être semblables à celles des preuves mathématiques. Ainsi, les étudiants en tirent de la connaissance sur comment faire des preuves.
%

\maketitle
\begin{frame}
\frametitle{Contexte}
\begin{itemize}
\item Utiliser des outils de preuves pour l'enseignement
\item Capacit\'es du langage, documentation, message d'erreur
\begin{itemize}
\item Inspiration : Waterproof
\item Exp\'eriences similaires sur Lean (Lean Verbose)
\end{itemize}
\item Le contenu des preuves joue un role primordial pas seulement leur pr\'esentation
\begin{itemize}
\item Plusieurs types de nombres, plusieurs versions d'op\'erateurs
\item Des coercions peuvent \^etre cach\'ees bloquant certaines op\'erations
\item La th\'eorie des types oblige les d\'eveloppeurs \`a d\'efinir des fonctions l\`a o\`u elles ne devraient pas \^etre d\'efinies.
\end{itemize}
\item Le typage aide les jeunes math\'ematiciens, mais pas le type des entiers naturels
\end{itemize}
\end{frame}
\begin{frame}
\frametitle{Caract\'eristiques des entiers naturels}
\begin{itemize}
\item Aspects positifs
\begin{itemize}
\item Un type inductif
\item Calcul par r\'eduction (plus rapide que la r\'e\'ecriture)
\item Les d\'efinitions r\'ecursives sont le plus souvent avec des entiers naturels
\end{itemize}
\item Aspects n\'egatifs
\begin{itemize}
\item La soustraction est \'etrange : la valeur de \(3-5\) est contre-intuitive
\item Le statut de la fonction/constructeur {\tt S} est un obstacle pour les \'etudiants
\item Dans Coq, {\tt S 4} et {\tt 5} sont interchangeables,
mais {\tt S x} et {\tt x + 1} ne le sont pas.
\item Le temps consacr\'e \`a l'apprentissage du pattern matching n'est pas consacr\'e aux math\'ematiques
\item La charge cognitive est trop importante
\end{itemize}
\end{itemize}
\end{frame}
\iffalse 
\begin{frame}
\frametitle{Les nombres dans l'esprit des d\'ebutants en math\'ematiques}
\begin{itemize}
\item D\`es l'\^age de 15 ans, les enfants connaissent probablement les nombres entiers, rationnels et r\'eels.

\item Le nombre \(3 - 5\) existe en tant que nombre, il n'est pas 0.
\item Le calcul de \(127 - 42\) donne un nombre naturel, \(3 - 5\) un nombre entier,
et \(1 / 3\) un rationnel
\item Le calcul de \(42 / 6\) donne un nombre naturel
\item Cette perception est {\em juste}, la respecter permet de gagner du temps
\end{itemize}
\end{frame}
\fi
\begin{frame}
\frametitle{Proposition}
\begin{itemize}
\item N'utiliser qu'un seul type de nombres : les nombres r\'eels
\begin{itemize}
\item Choisi pour \^etre intuitif pour les \'el\`eves \`a la fin du lyc\'ee
\item Incluant la relation d'ordre
\end{itemize}
\item Consid\'erer les autres types connus comme des sous-ensembles
\item Inclure les lois de stabilit\'e dans l'automatisation des preuves
\item Inspiration forte : l'approche PVS
\begin{itemize}
\item Cependant, PVS est trop agressif en mati\`ere d'automatisation pour l'\'education.
\end{itemize}
\item Entiers naturels, relatifs sont encore utilis\'es de mani\`ere cach\'ee
\end{itemize}
\end{frame}
\begin{frame}
\frametitle{Plan}
\begin{itemize}
\item Revue des usages des entiers naturels et relatifs
\item D\'efinir des sous-ensembles de \(\mathbb R\) pour les types inductifs
\item De \(\mathbb Z\) et \(\mathbb N\) \`a \(\mathbb R\) et vice-versa
\item Preuves ad hoc d'appartenance
\item D\'efinitions r\'ecursives et fonctions it\'er\'ees
\item Exemple pratique : autour de Fibonacci

\end{itemize}
\end{frame}
\begin{frame}
\frametitle{Utilisation des entiers naturels et relatifs}
\begin{itemize}
\item Une base pour les preuves par r\'ecurrence
\item D\'efinition d'une suite r\'ecursive
\item  It\'erer une op\'eration un certain nombre de fois \(f^n(k)\)
\item  La s\'equence \(0, \ldots, n\)

\item Indices pour les op\'erations it\'er\'ees \(\sum_{i=m}^{n} f(i)\)
\iffalse
\item  Indices pour les collections finies, % TODO  A_1 ...
\item Sp\'ecifique \`a Coq+Lean+Agda : formes normales des constructeurs comme cibles de la r\'eduction
\item \fi
\item  Dans les nombres r\'eels de Coq, les nombres 0, 1, ..., 37, ... s'appuient sur le type inductif des entiers relatifs pour la repr\'esentation. 
\begin{itemize}
\item En Coq, vous pouvez d\'efinir {\tt Zfact} comme un \'equivalent efficace de la factorielle et calculer 100!
\end{itemize}
\end{itemize}
\end{frame}

\begin{frame}
\frametitle{D\'efinir des sous-ensembles de \(\mathbb R\) pour les types inductifs}
\begin{itemize}
\item Approche inductive des pr\'edicats
\begin{itemize}
\item H\'eriter du principe de r\'ecurrence
\item Prouver l'existence d'un entier naturel ou relatif correspondant


\end{itemize}
\end{itemize}
\end{frame}
\begin{frame}[fragile]
\frametitle{Principe de r\'ecurrence en Coq}
\begin{alltt}
Require Import Reals.
Open Scope R_scope.

Inductive Rnat : R -> Prop :=
  Rnat0 : Rnat 0
| Rnat_succ : forall n, Rnat n -> Rnat (n + 1).
\end{alltt}
Principe de r\'ecurrence g\'en\'er\'e:
\begin{alltt}
nat_ind
     : forall P : R -> Prop,
       P 0 ->
       (forall n : R, Rnat n -> P n -> P (n + 1)) ->
       forall r : R, Rnat r -> P r 
\end{alltt}%bien insister
\end{frame}
\begin{frame}
\frametitle{De \(\mathbb N\) et \(\mathbb Z\) vers \(\mathbb R\) et vice-versa}
\begin{itemize}
\item Rappel : les types \(\mathbb N\) ({\tt nat}) et \(\mathbb Z\) ({\tt Z}),
ne doivent pas \^etre expos\'es
\item Les injections {\tt INR} et {\tt IZR} existent d\'ej\`a
\item Nouvelles fonctions {\tt IRN} et {\tt IRZ}
\item D\'efinissables \`a l'aide d'outils de logique classiques comme l'axiome de choix d'Hilbert ou la partie enti\`ere.

\end{itemize}
\end{frame}
\begin{frame}[fragile]
\frametitle{Typage d\'egrad\'e}
\begin{itemize}
\item Les lois de stabilit\'e fournissent des preuves automatisables de l'appartenance \`a un sous-ensemble.
\end{itemize}
\begin{small}
\begin{alltt}
Existing Class Rnat.

Lemma Rnat_add x y : Rnat x -> Rnat y -> Rnat (x + y).
Proof. ... Qed.

Lemma Rnat_mul x y : Rnat x -> Rnat y -> Rnat (x * y).
Proof. ... Qed.

Lemma Rnat_pos : Rnat (IZR (Z.pos _)).
Proof. ... Qed.

Existing Instances Rnat0 Rnat_succ Rnat_add Rnat_mul Rnat_pos.
\end{alltt}
\end{small}
\begin{itemize}
\item {\tt typeclasses eauto} ou {\tt exact \_.} r\'esoudra automatiquement
   {\tt Rnat x -> Rnat ((x + 2) * x)}.
\end{itemize}
\end{frame}
\begin{frame}
\frametitle{Preuves ad hoc d'appartenance}
\begin{itemize}
\item Quand \(n, m \in {\mathbb N}, m \leq n\), \((n - m) \in {\mathbb N}\) peut aussi \^etre prouv\'e
\item Cela n\'ecessite une preuve explicite
\item Probablement bon dans un contexte de formation pour les \'etudiants
\item Idem pour la division
\end{itemize}
\end{frame}
\begin{frame}
\frametitle{Fonctions r\'ecursives}
\begin{itemize}
\item Les suites r\'ecurrentes sont \'egalement un sujet d'introduction typique
\item \`A titre d'illustration, consid\'erons la suite de Fibonacci.
{\em La suite de Fibonacci est la fonction \(F\) telle que \(F_0= 0\), \(F_1=1\), et \(F_{n+2} = F_{n} + F_{n + 1}\) pour tout nombre naturel \(n\)}.
\item La preuve par r\'ecurrence et les \'equations de d\'efinition suffisent \`a
{\em \'etudier} une suite
\item Mais il faut encore la d\'efinir
\item Solution : d\'efinir une commande de d\'efinition r\'ecursive \`a l'aide d'ELPI
\end{itemize}
\end{frame}
\begin{frame}[fragile]
\frametitle{D\'efinition de fonctions r\'ecursives}
\begin{itemize}
\item On peut utiliser un {\em r\'ecurseur}, miroir du r\'ecurseur sur les entiers naturels
\item {\tt Rnat\_rec : ?A -> (R -> ?A -> ?A) -> R -> ?A}
\item La r\'ecursivit\'e d'ordre sup\'erieur \`a 1 peut \^etre mise en \oe{}uvre en utilisant des tuples de la bonne taille
\end{itemize}
\begin{small}
\begin{alltt}
(* fib 0 = 0  fib 1 = 1              *)
(* fib n = fib (n - 1) + fib (n - 2) *)

Definition fibr := Rnat_rec [0; 1]
   (fun n l => [nth 1 l 0; nth 0 l 0 + nth 1 l 0]).

Definition fib n := nth 0 (fibr n) 0. 
\end{alltt}
\end{small}%parler du temps de calcul gagné 
\end{frame}
\begin{frame}[fragile]
\frametitle{Meta-programmation d'une commande de d\'efinition r\'ecursive}
\begin{itemize}
\item La d\'efinition du transparent pr\'ec\'edent peut \^etre g\'en\'er\'ee
\item En prenant comme entr\'ee les \'equations (dans les commentaires)
\item Les r\'esultats de la d\'efinition sont en deux parties
\begin{itemize}
\item La fonction de type {\tt R -> R}
\item La preuve, l'\'enonc\'e logique de cette fonction
\end{itemize}
\end{itemize}
\begin{small}
\begin{verbatim}
Recursive (def fib such that
             fib 0 = 0 /\
             fib 1 = 1 /\
             forall n : R, Rnat (n - 2) ->
                fib n = fib (n - 2) + fib (n - 1)).
\end{verbatim}
\end{small}
\end{frame}
\begin{frame}
\frametitle{Calculer avec des nombres r\'eels}
\begin{itemize}
\item {\tt Compute 42 - 67.} donne une r\'eponse surprenante
\begin{itemize}
\item Des tas de {\tt R1}, {\tt +}, {\tt *} et de parenth\`eses.
\end{itemize}
\item {\tt Compute (42 - 67)\%Z.} donne la bonne r\'eponse
\begin{itemize}
\item  Sauf qu'elle n'a pas le bon type
\end{itemize}
\end{itemize}
\end{frame}
\begin{frame}[fragile]
\frametitle{Calculer avec des nombres r\'eels : notre solution}
\begin{itemize}
\item Nouvelle commande {\tt R\_compute}.
\item {\tt R\_compute (42 - 67).} r\'eussit et affiche {\tt -25}
\item {\tt R\_compute (fib (42 - 67)).} \'echoue!
\item {\tt R\_compute (fib 42) th\_name.} r\'eussit et enregistre la preuve de l'\'egalit\'e.
\iffalse
\begin{itemize}
\item En respectant le fait que {\tt fib} n'est d\'efini que pour des entr\'ees en nombres naturels
\end{itemize}
\fi
\item Impl\'ement\'e  en exploitant une correspondance entre les op\'erations \'el\'ementaires sur {\tt R}, {\tt Z} (avec preuves)
\item Programme en ELPI
\item Miroir automatique d'une d\'efinition r\'ecursive dans {\tt R} vers une d\'efinition dans {\tt Z}
\item Th\'eor\`eme de correction du processus miroir a une pr\'econdition {\tt Rnat} sur l'entr\'ee.
\end{itemize}
\end{frame}

\begin{frame}
   \frametitle{D\'emonstration}
  
   \end{frame}
\end{document}


